The photodiode experiment demonstrates that forward-biasing the diode impedes its ability to generate electrical energy from light. However, the data fails to demonstrate a significant difference between the reverse-bias and zero-bias cases. Possible explanations include the bias voltage needs to be larger to observe the photodiode supplying more power, the depletion region of the diode is already wide enough that biasing it does not increase absorption substantially, and the phototransistor may have slightly different properties from a true photodiode. Solar cells designs often employ a reverse-bias case to widen the depletion region, implying more experimentation is required to determine whether reverse or zero biasing is preferable (\ref{ref:packaging_pn_diode}). Given the current results, zero biasing is preferable because it does not require an external source voltage and produces essentially the same effect as the reverse bias case.
