\underline{Procedure}

For the fifth experiment in Electronics I Lab, the objective is to measure the minority carrier storage time of p-n junction diodes, the transient response, and verify that there is no involvement of minority carriers in Schottky diodes. The first part of the experiment consists of building a full-wave diode rectifier and observing the many voltage waveforms on the oscilloscope. A simple circuit is built with a waveform generator that has an internal resistance of 50Ω. The generator is connected to a rectifier p-n junction diode in series, while also having a resistor (R_L) in parallel with the oscilloscope. The transient response is observed from the circuit after applying voltage across the test diode by the pulse waveform. An input waveform and output waveform should be seen using the oscilloscope. The reverse bias current, forward bias current, and storage time are then measured while the minority carrier lifetime is observed on the side of the junction that is lightly doped. Lastly, there should be an observation of no rectifying function when the voltage waveform period becomes smaller than 2ts. Switch the rectifier p-n junction diode with a switching p-n junction diode and repeat the same steps. For the final circuit, replace the switching p-n junction diode with a Schottky diode and repeat the same steps. For the final part of the experiment a full-wave diode rectifier is built using 4 diodes that are connected to a function generator and a resistor (R). Set the parameters of the waveform generator to a voltage of 10 {V_{pp}} with a frequency of 1kHz, and set the resistor value to 1k\ohm. Record the input voltage {V_{in}} and measure the voltage across the load resistor. \\

