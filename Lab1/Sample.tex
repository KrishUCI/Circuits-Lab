\documentclass[a4paper,10pt]{article}
\usepackage{siunitx}
\usepackage{spreadtab}
% url package is used for embedding URLs.
\usepackage{url}
% Start writing your document!
\begin{document}

% \\ creates a new line.
\underline{Procedure}

For the Electronics Lab course, the multimeter model that is used is the 34405A. This multimeter allows us to help measure features such as either true RMS AC or DC voltage and as well as true RMS AC or DC current. The range of the voltage on the multimeter goes from 10mV to 1000V, while the range for the current goes from 10mA to 10A. The input impedance of the multimeter is around 1M\si{\ohm} +/- 2\% for AC voltage while the input resistance for DC voltage is listed around 10M\si{\ohm} +/- 2\%. The circuit in our lab is a simple series circuit, therefore the voltmeter acts as a voltage divider.\\
% Use $ to create a math equation. \centerline centers the text in brackets.
\centerline{ $ V_VM = V_Source(\frac{R_Vm}{R_VM + R} $}
\centerline{ $ \frac{\hbar^2}{2m} \nabla^2\Psi + \Psi V = j\hbar\frac{ \partial \Psi }{ \partial t } $ }

LaTeX also lets you create tables:
\begin{table}[h!]
\centering
\caption{My first LaTeX table}
\label{tab:table1}
\begin{spreadtab}{{tabular}{|c|c|c|}}
	\hline
	@Frequency [ Hz ] & @Scope Voltage at R = 1k$\Omega$ [ V ] & @Scope Voltage at R = 1M$\Omega$ [ V ] \\
	\hline
	1kHz & 10.25 & 5.00 \\
	1MHz & 9.31 & 0.1 \\
	\hline
\end{spreadtab}
\end{table}

% Note you will need to compile LaTeX twice for \ref to work. I haven't read this in its entirely yet, but the following link explains why: https://tex.stackexchange.com/questions/111280/understanding-how-references-and-labels-work
A really critical feature of LaTeX for us is that we can refer to tables by reference as opposed to by the name directly. Look in the LaTeX source for how I refer to \ref{tab:table1} without actually explicitly writing out "Table 1" or what have you. This way, we can simply give our tables variable names, and the numbering will be automatically done for us. This automated numbering scheme also extends to the text. This can also be extended to page numbers. See this link for more details: \url{https://en.wikibooks.org/wiki/LaTeX/Labels_and_Cross-referencing}.

\end{document}
