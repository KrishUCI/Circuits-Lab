Both the pn-junction diode and Schottky diode yielded waveforms that agree well with theory. In analyzing the two types of diodes, the advantages and disadvantages of each are made clear. The Schottky diode is observed to have a nonexistent storage time whereas the pn-junction diode is observed to have a storage time of approximately $4\mu s$. The very apparent delay in rectification from the storage time of the pn-junction diode makes that type of diode not suited for high frequency applications. The Schottky diode, however, is very commonly used in high frequency applications like RF. What is not shown in this experiment is that Schottky diodes have significantly lower reverse breakdown voltages than pn-junction diodes  (\ref{ref:schottky2}), therefore Schottky diodes may not be suitable for high voltage environments. 
It should also be noted that the diode's 10\% error or so in Table (1) is likely a result of nonidealities in the manufacturing process of both the diode and the resistor. The resistor used in the measurement likely deviates from the $300\Omega$ value, and the diode has some internal resistance as well. Moreover, the diode's properties, such as $V_{on}$ in equations (5) and (6) are also subject to variations in the manufacturing process, such as in doping concentrations. The important part of the experiment is the shape of the waveform in figure (4). Better characterization of the properties of the resistor and the diode are required for more accurate results.
