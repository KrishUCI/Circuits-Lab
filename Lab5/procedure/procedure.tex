The objective of the experiment is to measure the minority carrier storage time of p-n junction diodes, referred to as $t_s$, the transient response, and to note differences in the transient response of Schottky diodes. The first part of the experiment consists of building a full-wave diode rectifier and observing the many voltage waveforms on the oscilloscope. A simple circuit is built with a waveform generator that has an internal resistance of 50\si{\ohm}. The generator is connected to a rectifier p-n junction diode in series, while also having a resistor R\textsubscript{L} in parallel with the oscilloscope. The transient response is observed from the circuit after applying a pulse waveform across the test diode. An input waveform and output waveform should be seen using the oscilloscope. The reverse bias current, forward bias current, and storage time are then measured while the minority carrier lifetime $\tau$ is observed on the side of the junction that is lightly doped. Lastly, there should be an observation of no rectifying function when the voltage waveform period becomes smaller than $2t_s$. Switch the rectifier p-n junction diode with a switching p-n junction diode and repeat the same steps. For the final circuit, replace the switching p-n junction diode with a Schottky diode and repeat the same steps. For the final part of the experiment a full-wave diode rectifier is built using 4 diodes that are connected to a function generator and a resistor (R). The parameters of the waveform generator are set to a voltage of 10\si{\volt}pp with a frequency of 1\si{\kilo\hertz}, and set the resistor value to 300\si{\ohm}. Record the input voltage V\textsubscript{in} and measure the voltage across the load resistor. \\
