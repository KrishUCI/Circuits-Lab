\underline{Procedure}

The objective of this lab is to observe basic semiconductor characteristics of a wafer after a hot-probe is placed on it. The hot probe helps determine if the semiconductor is a n-type or a p-type. The resistivity is also measured using a four-point probe. The experiment begins by heating up a soldering iron and then placing it on a silicon wafer. Two probes are placed on the wafer, one near the soldering iron and the other far away. The voltage between the hot and cold probes is then measured. The positive and negative probes are then flipped to measure the voltage between the two probes again. After completing the first part with the soldering iron, the final part deals with measuring the resistivity after a four-point probe is placed on the silicon wafer. \\


