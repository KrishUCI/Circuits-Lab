The hot probe technique is utilized to determine the type of unknown semiconductor material, pieces of silicon wafer. The first wafer tested was a large piece of silicon wafer approximately once centimeter in width and three centimeters in length. The specific piece of silicon wafer tested is shown in the image below:

\begin{figure}[h!]
	\centering
	\includegraphics[scale=0.1]{"../images/wafer".JPG}
	\caption{Image of First Silicon Wafer Tested with Hot Probe Technique}
	\label{fig:wafer}
\end{figure}

In the hot probe technique, a temperature gradient is introduced to the silicon wafer through contact with a hot probe, in this case a soldering iron, to one end of the wafer. Then, a digital voltmeter with high impedance is used to measure the voltage across the wafer from the hot end to the cool end (positive terminal to the hot end, negative terminal to the cool end). The sign of the voltage indicates the type of the semiconductor material. The circuit schematic below portrays the setup of the hot probe technique:

\begin{figure}[h!]
	\centering
	\includegraphics[scale=0.7]{"../images/hot_probe_schematic".PNG}
	\caption{Circuit Schematic of Hot Probe Technique}
	\label{fig:hot_probe_schematic}
\end{figure}

The hot probe and positive terminal of the digital voltmeter is first introduced to one end of the silicon wafer and is set there for approximately 30 seconds to generate the temperature gradient across the wafer. Then, the negative terminal of the voltmeter, cool due to no prior contact with the heated wafer and hot probe is introduced to the other end of the wafer. With to the increased temperature difference between both ends of the wafer as a result of this method, a more optimal voltage measurement can be achieved.

The following voltages are indicated by the digital voltmeter with $\pm 1 V$ fluctuations in each case:

\begin{table}[h!]
	\centering
	\caption{Hot Probe Measurements for First Wafer}
	\label{tab:hot_probe_measurements_p_type}
	\csvautotabular{../tables/hot_probe_measurements_p_type.csv}
\end{table}

The high negative voltages in Table \ref{tab:hot_probe_measurements_p_type} clearly indicate that the first wafer tested is a p-type semiconductor. This is because the temperature gradient in the causes charge carriers to move towards the cooler side of the wafer, thus inducing a current in the material and ultimately a voltage difference across the semiconductor. If the semiconductor material is p-type, then the charge carriers, holes, will migrate from the hot end to the cool end, leaving ionized acceptors with negative charge at the hot end. This polarizes the material so that the hot end is negatively charged and the cool end is positively charged. Measuring voltage from the hot end to the cool end in this case would then yield a negative result. For n-type semiconductors, the charge carriers, electrons, also migrate from the hot end to cool end, which then leaves ionized donors with positive charge at the hot end. This polarizes the material so that the hot end is positively charged and the cool end is negatively charged. Measuring voltage from the hot end to the cool end in this case would then yield a positive result.

During testing of p-type silicon wafers, voltage measurements across the ends of the wafers were very noisy. Below are the measurements taken for two wafers p-type wafers:

\begin{table}[h!]
	\centering
	\caption{Hot Probe Measurements for P-type Wafers}
	\label{tab:hot_probe_measurements_n_type}
	\csvautotabular{../tables/hot_probe_measurements_n_type.csv}
\end{table}