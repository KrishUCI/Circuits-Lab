In analyzing the low-pass and high-pass filter setups with sinusoidal inputs, a peak-to-peak voltage of 10 V is used. Peak-to-peak voltages of the output signal at a range of frequencies are used to construct the Bode plots for the low-pass and high-pass filters. A resistor with R = 10 k$\Omega$, a capacitor with C = 10 nF, and a function generator with 50 $\Omega$ internal resistance is used to build the filters.

\subsection{Low-Pass Filter}

The output voltage signal for the low-pass filter is taken across the capacitor in the following schematic:

\begin{figure}[h!]
	\centering
	\includegraphics[scale=0.6]{"../images/LPF_schematic".PNG}
	\caption{Low-pass Filter Circuit Schematic}
	\label{fig:LPF_Schematic}
\end{figure}

The transfer function of this filter is the following where the internal resistance of the function generator is neglected due to its low value:

\begin{equation}
\label{eq:lpf_transfer}
H_{LPF}(s) = \frac{1}{1+sRC}
\end{equation}

The following output voltages are recorded at the frequencies tested below along with the calculated gain in decibels:

\FloatBarrier

\begin{table}[h!]
	\centering
	\caption{Low-Pass Filter Output Voltages at Various Frequencies}
	\label{tab:lpf_vout}
	\csvautotabular{../tables/vpp_freq_lpf.csv}
\end{table}

\FloatBarrier

The gain values above are calculated using the definition of decibel scale taking the ratio of output and input power. The equation simplifies to this:

\begin{equation}
\label{eq:decibels}
Gain = 10 log(\frac{P_{out}}{P_{in}}) = 10 log (\frac{V_{out}}{ V_{in}})^2 = 20 log (\frac{V_{out}}{ V_{in}})
\end{equation}

Using the gain values, the following bode plot for the low-pass filter is generated:

\FloatBarrier

\begin{figure}[h!]
	\centering
	\includegraphics[scale=0.5]{"../images/LPF_Bode".PNG}
	\caption{Low-pass Filter Circuit Bode Plot}
	\label{fig:LPF_Bode}
\end{figure}

\FloatBarrier

The cutoff frequency can be obtained by finding the value of the frequency in which the system outputs a -3 dB gain. By inspection of the Bode plot above, the corner frequency of the low-pass filter is 18 kHz.

Theoretically, the cutoff frequency of the filter occurs when the magnitude of the transfer function of the filter is $\frac{1}{\sqrt{2}}$. From \ref{eq:lpf_transfer}, the theoretical cutoff frequency can be obtained:

\begin{equation}
\label{eq:lpf_cutoff}
\begin{gathered}
|H_{LPF}(s)| = |\frac{1}{1+sRC}| = \frac{1}{\sqrt{2}}\\
\frac{1}{\sqrt{1^2 + (\omega_{c} RC)^2}} = \frac{1}{\sqrt{2}}\\
1^2 + (\omega_{c} RC)^2 = 2\\
(\omega_{c} RC)^2 = 1\\
\omega_{c} = \frac{1}{RC} = \frac{1}{(10^5 \Omega)(10^{-9} F)}\\
\omega_{c} = 10^5\\
f_{c}  = \frac{\omega_{c}}{2\pi} = 15.915 k\Omega
\end{gathered}
\end{equation}

Compared to the measured cutoff frequency from \ref{fig:LPF_Bode}, the following error is calculated:

\begin{equation}
\label{eq:lpf_cutoff_error}
|\frac{18 - 15.915}{15.915}| * 100\% = 13.1\%
\end{equation}


\subsection{High-Pass Filter}

The output voltage signal for the low-pass filter is taken across the capacitor in the following schematic:

\begin{figure}[h!]
	\centering
	\includegraphics[scale=0.6]{"../images/HPF_schematic".PNG}
	\caption{High-pass Filter Circuit Schematic}
	\label{fig:HPF_Schematic}
\end{figure}

The transfer function of this filter is the following where again, the resistance of the function generator is ignored:

\begin{equation}
\label{eq:hpf_transfer}
H_{HPF}(s) = \frac{sRC}{1+sRC}
\end{equation}

The following output voltages are recorded at the frequencies tested below along with the calculated gain in decibels calculated using \ref{eq:decibels}:

\FloatBarrier

\begin{table}[h!]
	\centering
	\caption{High-Pass Filter Output Voltages at Various Frequencies}
	\label{tab:hpf_vout}
	\csvautotabular{../tables/vpp_freq_hpf.csv}
\end{table}

\FloatBarrier

The gain values produce the following Bode plot:

\FloatBarrier

\begin{figure}[h!]
	\centering
	\includegraphics[scale=0.5]{"../images/HPF_Bode".PNG}
	\caption{High-pass Filter Circuit Bode Plot}
	\label{fig:HPF_Bode}
\end{figure}

\FloatBarrier

Again, by inspection, the cutoff frequency of the high-pass filter is 16 kHz.

Theoretically, the cutoff frequency can be calculated as such using the magnitude of \ref{eq:hpf_transfer}:

\begin{equation}
\label{eq:hpf_cutoff}
\begin{gathered}
|H_{HPF}(s)| = |\frac{sRC}{1+sRC}| = \frac{1}{\sqrt{2}}\\
\frac{\omega_{c}RC}{\sqrt{1^2 + (\omega_{c} RC)^2}} = \frac{1}{\sqrt{2}}\\
\frac{(\omega_{c}RC)^2}{1^2 + (\omega_{c} RC)^2} = \frac{1}{2}\\
2(\omega_{c} RC)^2 = 1^2 + (\omega_{c} RC)^2\\
(\omega_{c} RC)^2 = 1\\
\omega_{c} = \frac{1}{RC} = \frac{1}{(10^5 \Omega)(10^{-9} F)}\\
\omega_{c} = 10^5\\
f_{c}  = \frac{\omega_{c}}{2\pi} = 15.915 k\Omega
\end{gathered}
\end{equation}

Compared to the measured cutoff frequency from \ref{fig:HPF_Bode}, the following error is calculated:

\begin{equation}
\label{eq:lpf_cutoff_error}
|\frac{16 - 15.915}{15.915}| * 100\% = 0.534\%
\end{equation}
