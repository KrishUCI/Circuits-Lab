The high-pass and low-pass filters both produce impulse responses that are relatively close to what theory predicts. Regarding the low-pass filter, the capacitor charges suddenly during the pulse, and the voltage gradually decays from this peak after the pulse ends due to the discharging of the capacitor. In the high-pass filter's case, the voltage over the resistor surges because the supply drives massive current through it to charge the capacitor. However, after the pulse, the polarity of the current flips due to the discharging of the capacitor. The voltage becomes negative, and its magnitude decays as the capacitor discharges. The accuracy of the measurements is limited by the ability to measure $\tau$ values from the oscilloscope. In order to acquire better measurements, the power supply needs to have a higher voltage, wider pulse width, and lower pulse frequency. More generally, the physical limitation of the experiment is that true Dirac deltas do not exist in practice since the pulse always has a finite width and a noninfinite height. For as long as this holds true, the experiment deviates from theoretical predictions.
