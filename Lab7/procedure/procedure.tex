The objective of this lab is to understand the use of the bipolar-junction transistor (BJT) in circuits, in particular the common-emitter amplifier, also known as the inverter. First, the circuit is tested with a pulse input. The inverting behavior as well as the transients are observed, and the time delay, rise time, and fall time are measured. This experiment demonstrates the switching capabilities of the BJT. Next, the BJT is used as an amplifier for sinusoidal input signals. An input signal at 10kHz is applied, and the input voltage is increased until the output voltage starts to clamp. The gain is then measured. After this, the upper cutoff frequency as well as the gain at this frequency are determined. Lastly, a square wave input at the cutoff frequency is tested and its output observed. \\
